
% Default to the notebook output style

    


% Inherit from the specified cell style.




    
\documentclass[11pt]{article}

    
    
    \usepackage[T1]{fontenc}
    % Nicer default font (+ math font) than Computer Modern for most use cases
    \usepackage{mathpazo}

    % Basic figure setup, for now with no caption control since it's done
    % automatically by Pandoc (which extracts ![](path) syntax from Markdown).
    \usepackage{graphicx}
    % We will generate all images so they have a width \maxwidth. This means
    % that they will get their normal width if they fit onto the page, but
    % are scaled down if they would overflow the margins.
    \makeatletter
    \def\maxwidth{\ifdim\Gin@nat@width>\linewidth\linewidth
    \else\Gin@nat@width\fi}
    \makeatother
    \let\Oldincludegraphics\includegraphics
    % Set max figure width to be 80% of text width, for now hardcoded.
    \renewcommand{\includegraphics}[1]{\Oldincludegraphics[width=.8\maxwidth]{#1}}
    % Ensure that by default, figures have no caption (until we provide a
    % proper Figure object with a Caption API and a way to capture that
    % in the conversion process - todo).
    \usepackage{caption}
    \DeclareCaptionLabelFormat{nolabel}{}
    \captionsetup{labelformat=nolabel}

    \usepackage{adjustbox} % Used to constrain images to a maximum size 
    \usepackage{xcolor} % Allow colors to be defined
    \usepackage{enumerate} % Needed for markdown enumerations to work
    \usepackage{geometry} % Used to adjust the document margins
    \usepackage{amsmath} % Equations
    \usepackage{amssymb} % Equations
    \usepackage{textcomp} % defines textquotesingle
    % Hack from http://tex.stackexchange.com/a/47451/13684:
    \AtBeginDocument{%
        \def\PYZsq{\textquotesingle}% Upright quotes in Pygmentized code
    }
    \usepackage{upquote} % Upright quotes for verbatim code
    \usepackage{eurosym} % defines \euro
    \usepackage[mathletters]{ucs} % Extended unicode (utf-8) support
    \usepackage[utf8x]{inputenc} % Allow utf-8 characters in the tex document
    \usepackage{fancyvrb} % verbatim replacement that allows latex
    \usepackage{grffile} % extends the file name processing of package graphics 
                         % to support a larger range 
    % The hyperref package gives us a pdf with properly built
    % internal navigation ('pdf bookmarks' for the table of contents,
    % internal cross-reference links, web links for URLs, etc.)
    \usepackage{hyperref}
    \usepackage{longtable} % longtable support required by pandoc >1.10
    \usepackage{booktabs}  % table support for pandoc > 1.12.2
    \usepackage[inline]{enumitem} % IRkernel/repr support (it uses the enumerate* environment)
    \usepackage[normalem]{ulem} % ulem is needed to support strikethroughs (\sout)
                                % normalem makes italics be italics, not underlines
    

    
    
    % Colors for the hyperref package
    \definecolor{urlcolor}{rgb}{0,.145,.698}
    \definecolor{linkcolor}{rgb}{.71,0.21,0.01}
    \definecolor{citecolor}{rgb}{.12,.54,.11}

    % ANSI colors
    \definecolor{ansi-black}{HTML}{3E424D}
    \definecolor{ansi-black-intense}{HTML}{282C36}
    \definecolor{ansi-red}{HTML}{E75C58}
    \definecolor{ansi-red-intense}{HTML}{B22B31}
    \definecolor{ansi-green}{HTML}{00A250}
    \definecolor{ansi-green-intense}{HTML}{007427}
    \definecolor{ansi-yellow}{HTML}{DDB62B}
    \definecolor{ansi-yellow-intense}{HTML}{B27D12}
    \definecolor{ansi-blue}{HTML}{208FFB}
    \definecolor{ansi-blue-intense}{HTML}{0065CA}
    \definecolor{ansi-magenta}{HTML}{D160C4}
    \definecolor{ansi-magenta-intense}{HTML}{A03196}
    \definecolor{ansi-cyan}{HTML}{60C6C8}
    \definecolor{ansi-cyan-intense}{HTML}{258F8F}
    \definecolor{ansi-white}{HTML}{C5C1B4}
    \definecolor{ansi-white-intense}{HTML}{A1A6B2}

    % commands and environments needed by pandoc snippets
    % extracted from the output of `pandoc -s`
    \providecommand{\tightlist}{%
      \setlength{\itemsep}{0pt}\setlength{\parskip}{0pt}}
    \DefineVerbatimEnvironment{Highlighting}{Verbatim}{commandchars=\\\{\}}
    % Add ',fontsize=\small' for more characters per line
    \newenvironment{Shaded}{}{}
    \newcommand{\KeywordTok}[1]{\textcolor[rgb]{0.00,0.44,0.13}{\textbf{{#1}}}}
    \newcommand{\DataTypeTok}[1]{\textcolor[rgb]{0.56,0.13,0.00}{{#1}}}
    \newcommand{\DecValTok}[1]{\textcolor[rgb]{0.25,0.63,0.44}{{#1}}}
    \newcommand{\BaseNTok}[1]{\textcolor[rgb]{0.25,0.63,0.44}{{#1}}}
    \newcommand{\FloatTok}[1]{\textcolor[rgb]{0.25,0.63,0.44}{{#1}}}
    \newcommand{\CharTok}[1]{\textcolor[rgb]{0.25,0.44,0.63}{{#1}}}
    \newcommand{\StringTok}[1]{\textcolor[rgb]{0.25,0.44,0.63}{{#1}}}
    \newcommand{\CommentTok}[1]{\textcolor[rgb]{0.38,0.63,0.69}{\textit{{#1}}}}
    \newcommand{\OtherTok}[1]{\textcolor[rgb]{0.00,0.44,0.13}{{#1}}}
    \newcommand{\AlertTok}[1]{\textcolor[rgb]{1.00,0.00,0.00}{\textbf{{#1}}}}
    \newcommand{\FunctionTok}[1]{\textcolor[rgb]{0.02,0.16,0.49}{{#1}}}
    \newcommand{\RegionMarkerTok}[1]{{#1}}
    \newcommand{\ErrorTok}[1]{\textcolor[rgb]{1.00,0.00,0.00}{\textbf{{#1}}}}
    \newcommand{\NormalTok}[1]{{#1}}
    
    % Additional commands for more recent versions of Pandoc
    \newcommand{\ConstantTok}[1]{\textcolor[rgb]{0.53,0.00,0.00}{{#1}}}
    \newcommand{\SpecialCharTok}[1]{\textcolor[rgb]{0.25,0.44,0.63}{{#1}}}
    \newcommand{\VerbatimStringTok}[1]{\textcolor[rgb]{0.25,0.44,0.63}{{#1}}}
    \newcommand{\SpecialStringTok}[1]{\textcolor[rgb]{0.73,0.40,0.53}{{#1}}}
    \newcommand{\ImportTok}[1]{{#1}}
    \newcommand{\DocumentationTok}[1]{\textcolor[rgb]{0.73,0.13,0.13}{\textit{{#1}}}}
    \newcommand{\AnnotationTok}[1]{\textcolor[rgb]{0.38,0.63,0.69}{\textbf{\textit{{#1}}}}}
    \newcommand{\CommentVarTok}[1]{\textcolor[rgb]{0.38,0.63,0.69}{\textbf{\textit{{#1}}}}}
    \newcommand{\VariableTok}[1]{\textcolor[rgb]{0.10,0.09,0.49}{{#1}}}
    \newcommand{\ControlFlowTok}[1]{\textcolor[rgb]{0.00,0.44,0.13}{\textbf{{#1}}}}
    \newcommand{\OperatorTok}[1]{\textcolor[rgb]{0.40,0.40,0.40}{{#1}}}
    \newcommand{\BuiltInTok}[1]{{#1}}
    \newcommand{\ExtensionTok}[1]{{#1}}
    \newcommand{\PreprocessorTok}[1]{\textcolor[rgb]{0.74,0.48,0.00}{{#1}}}
    \newcommand{\AttributeTok}[1]{\textcolor[rgb]{0.49,0.56,0.16}{{#1}}}
    \newcommand{\InformationTok}[1]{\textcolor[rgb]{0.38,0.63,0.69}{\textbf{\textit{{#1}}}}}
    \newcommand{\WarningTok}[1]{\textcolor[rgb]{0.38,0.63,0.69}{\textbf{\textit{{#1}}}}}
    
    
    % Define a nice break command that doesn't care if a line doesn't already
    % exist.
    \def\br{\hspace*{\fill} \\* }
    % Math Jax compatability definitions
    \def\gt{>}
    \def\lt{<}
    % Document parameters
    \title{Tail Risk Analysis}
    
    
    

    % Pygments definitions
    
\makeatletter
\def\PY@reset{\let\PY@it=\relax \let\PY@bf=\relax%
    \let\PY@ul=\relax \let\PY@tc=\relax%
    \let\PY@bc=\relax \let\PY@ff=\relax}
\def\PY@tok#1{\csname PY@tok@#1\endcsname}
\def\PY@toks#1+{\ifx\relax#1\empty\else%
    \PY@tok{#1}\expandafter\PY@toks\fi}
\def\PY@do#1{\PY@bc{\PY@tc{\PY@ul{%
    \PY@it{\PY@bf{\PY@ff{#1}}}}}}}
\def\PY#1#2{\PY@reset\PY@toks#1+\relax+\PY@do{#2}}

\expandafter\def\csname PY@tok@w\endcsname{\def\PY@tc##1{\textcolor[rgb]{0.73,0.73,0.73}{##1}}}
\expandafter\def\csname PY@tok@c\endcsname{\let\PY@it=\textit\def\PY@tc##1{\textcolor[rgb]{0.25,0.50,0.50}{##1}}}
\expandafter\def\csname PY@tok@cp\endcsname{\def\PY@tc##1{\textcolor[rgb]{0.74,0.48,0.00}{##1}}}
\expandafter\def\csname PY@tok@k\endcsname{\let\PY@bf=\textbf\def\PY@tc##1{\textcolor[rgb]{0.00,0.50,0.00}{##1}}}
\expandafter\def\csname PY@tok@kp\endcsname{\def\PY@tc##1{\textcolor[rgb]{0.00,0.50,0.00}{##1}}}
\expandafter\def\csname PY@tok@kt\endcsname{\def\PY@tc##1{\textcolor[rgb]{0.69,0.00,0.25}{##1}}}
\expandafter\def\csname PY@tok@o\endcsname{\def\PY@tc##1{\textcolor[rgb]{0.40,0.40,0.40}{##1}}}
\expandafter\def\csname PY@tok@ow\endcsname{\let\PY@bf=\textbf\def\PY@tc##1{\textcolor[rgb]{0.67,0.13,1.00}{##1}}}
\expandafter\def\csname PY@tok@nb\endcsname{\def\PY@tc##1{\textcolor[rgb]{0.00,0.50,0.00}{##1}}}
\expandafter\def\csname PY@tok@nf\endcsname{\def\PY@tc##1{\textcolor[rgb]{0.00,0.00,1.00}{##1}}}
\expandafter\def\csname PY@tok@nc\endcsname{\let\PY@bf=\textbf\def\PY@tc##1{\textcolor[rgb]{0.00,0.00,1.00}{##1}}}
\expandafter\def\csname PY@tok@nn\endcsname{\let\PY@bf=\textbf\def\PY@tc##1{\textcolor[rgb]{0.00,0.00,1.00}{##1}}}
\expandafter\def\csname PY@tok@ne\endcsname{\let\PY@bf=\textbf\def\PY@tc##1{\textcolor[rgb]{0.82,0.25,0.23}{##1}}}
\expandafter\def\csname PY@tok@nv\endcsname{\def\PY@tc##1{\textcolor[rgb]{0.10,0.09,0.49}{##1}}}
\expandafter\def\csname PY@tok@no\endcsname{\def\PY@tc##1{\textcolor[rgb]{0.53,0.00,0.00}{##1}}}
\expandafter\def\csname PY@tok@nl\endcsname{\def\PY@tc##1{\textcolor[rgb]{0.63,0.63,0.00}{##1}}}
\expandafter\def\csname PY@tok@ni\endcsname{\let\PY@bf=\textbf\def\PY@tc##1{\textcolor[rgb]{0.60,0.60,0.60}{##1}}}
\expandafter\def\csname PY@tok@na\endcsname{\def\PY@tc##1{\textcolor[rgb]{0.49,0.56,0.16}{##1}}}
\expandafter\def\csname PY@tok@nt\endcsname{\let\PY@bf=\textbf\def\PY@tc##1{\textcolor[rgb]{0.00,0.50,0.00}{##1}}}
\expandafter\def\csname PY@tok@nd\endcsname{\def\PY@tc##1{\textcolor[rgb]{0.67,0.13,1.00}{##1}}}
\expandafter\def\csname PY@tok@s\endcsname{\def\PY@tc##1{\textcolor[rgb]{0.73,0.13,0.13}{##1}}}
\expandafter\def\csname PY@tok@sd\endcsname{\let\PY@it=\textit\def\PY@tc##1{\textcolor[rgb]{0.73,0.13,0.13}{##1}}}
\expandafter\def\csname PY@tok@si\endcsname{\let\PY@bf=\textbf\def\PY@tc##1{\textcolor[rgb]{0.73,0.40,0.53}{##1}}}
\expandafter\def\csname PY@tok@se\endcsname{\let\PY@bf=\textbf\def\PY@tc##1{\textcolor[rgb]{0.73,0.40,0.13}{##1}}}
\expandafter\def\csname PY@tok@sr\endcsname{\def\PY@tc##1{\textcolor[rgb]{0.73,0.40,0.53}{##1}}}
\expandafter\def\csname PY@tok@ss\endcsname{\def\PY@tc##1{\textcolor[rgb]{0.10,0.09,0.49}{##1}}}
\expandafter\def\csname PY@tok@sx\endcsname{\def\PY@tc##1{\textcolor[rgb]{0.00,0.50,0.00}{##1}}}
\expandafter\def\csname PY@tok@m\endcsname{\def\PY@tc##1{\textcolor[rgb]{0.40,0.40,0.40}{##1}}}
\expandafter\def\csname PY@tok@gh\endcsname{\let\PY@bf=\textbf\def\PY@tc##1{\textcolor[rgb]{0.00,0.00,0.50}{##1}}}
\expandafter\def\csname PY@tok@gu\endcsname{\let\PY@bf=\textbf\def\PY@tc##1{\textcolor[rgb]{0.50,0.00,0.50}{##1}}}
\expandafter\def\csname PY@tok@gd\endcsname{\def\PY@tc##1{\textcolor[rgb]{0.63,0.00,0.00}{##1}}}
\expandafter\def\csname PY@tok@gi\endcsname{\def\PY@tc##1{\textcolor[rgb]{0.00,0.63,0.00}{##1}}}
\expandafter\def\csname PY@tok@gr\endcsname{\def\PY@tc##1{\textcolor[rgb]{1.00,0.00,0.00}{##1}}}
\expandafter\def\csname PY@tok@ge\endcsname{\let\PY@it=\textit}
\expandafter\def\csname PY@tok@gs\endcsname{\let\PY@bf=\textbf}
\expandafter\def\csname PY@tok@gp\endcsname{\let\PY@bf=\textbf\def\PY@tc##1{\textcolor[rgb]{0.00,0.00,0.50}{##1}}}
\expandafter\def\csname PY@tok@go\endcsname{\def\PY@tc##1{\textcolor[rgb]{0.53,0.53,0.53}{##1}}}
\expandafter\def\csname PY@tok@gt\endcsname{\def\PY@tc##1{\textcolor[rgb]{0.00,0.27,0.87}{##1}}}
\expandafter\def\csname PY@tok@err\endcsname{\def\PY@bc##1{\setlength{\fboxsep}{0pt}\fcolorbox[rgb]{1.00,0.00,0.00}{1,1,1}{\strut ##1}}}
\expandafter\def\csname PY@tok@kc\endcsname{\let\PY@bf=\textbf\def\PY@tc##1{\textcolor[rgb]{0.00,0.50,0.00}{##1}}}
\expandafter\def\csname PY@tok@kd\endcsname{\let\PY@bf=\textbf\def\PY@tc##1{\textcolor[rgb]{0.00,0.50,0.00}{##1}}}
\expandafter\def\csname PY@tok@kn\endcsname{\let\PY@bf=\textbf\def\PY@tc##1{\textcolor[rgb]{0.00,0.50,0.00}{##1}}}
\expandafter\def\csname PY@tok@kr\endcsname{\let\PY@bf=\textbf\def\PY@tc##1{\textcolor[rgb]{0.00,0.50,0.00}{##1}}}
\expandafter\def\csname PY@tok@bp\endcsname{\def\PY@tc##1{\textcolor[rgb]{0.00,0.50,0.00}{##1}}}
\expandafter\def\csname PY@tok@fm\endcsname{\def\PY@tc##1{\textcolor[rgb]{0.00,0.00,1.00}{##1}}}
\expandafter\def\csname PY@tok@vc\endcsname{\def\PY@tc##1{\textcolor[rgb]{0.10,0.09,0.49}{##1}}}
\expandafter\def\csname PY@tok@vg\endcsname{\def\PY@tc##1{\textcolor[rgb]{0.10,0.09,0.49}{##1}}}
\expandafter\def\csname PY@tok@vi\endcsname{\def\PY@tc##1{\textcolor[rgb]{0.10,0.09,0.49}{##1}}}
\expandafter\def\csname PY@tok@vm\endcsname{\def\PY@tc##1{\textcolor[rgb]{0.10,0.09,0.49}{##1}}}
\expandafter\def\csname PY@tok@sa\endcsname{\def\PY@tc##1{\textcolor[rgb]{0.73,0.13,0.13}{##1}}}
\expandafter\def\csname PY@tok@sb\endcsname{\def\PY@tc##1{\textcolor[rgb]{0.73,0.13,0.13}{##1}}}
\expandafter\def\csname PY@tok@sc\endcsname{\def\PY@tc##1{\textcolor[rgb]{0.73,0.13,0.13}{##1}}}
\expandafter\def\csname PY@tok@dl\endcsname{\def\PY@tc##1{\textcolor[rgb]{0.73,0.13,0.13}{##1}}}
\expandafter\def\csname PY@tok@s2\endcsname{\def\PY@tc##1{\textcolor[rgb]{0.73,0.13,0.13}{##1}}}
\expandafter\def\csname PY@tok@sh\endcsname{\def\PY@tc##1{\textcolor[rgb]{0.73,0.13,0.13}{##1}}}
\expandafter\def\csname PY@tok@s1\endcsname{\def\PY@tc##1{\textcolor[rgb]{0.73,0.13,0.13}{##1}}}
\expandafter\def\csname PY@tok@mb\endcsname{\def\PY@tc##1{\textcolor[rgb]{0.40,0.40,0.40}{##1}}}
\expandafter\def\csname PY@tok@mf\endcsname{\def\PY@tc##1{\textcolor[rgb]{0.40,0.40,0.40}{##1}}}
\expandafter\def\csname PY@tok@mh\endcsname{\def\PY@tc##1{\textcolor[rgb]{0.40,0.40,0.40}{##1}}}
\expandafter\def\csname PY@tok@mi\endcsname{\def\PY@tc##1{\textcolor[rgb]{0.40,0.40,0.40}{##1}}}
\expandafter\def\csname PY@tok@il\endcsname{\def\PY@tc##1{\textcolor[rgb]{0.40,0.40,0.40}{##1}}}
\expandafter\def\csname PY@tok@mo\endcsname{\def\PY@tc##1{\textcolor[rgb]{0.40,0.40,0.40}{##1}}}
\expandafter\def\csname PY@tok@ch\endcsname{\let\PY@it=\textit\def\PY@tc##1{\textcolor[rgb]{0.25,0.50,0.50}{##1}}}
\expandafter\def\csname PY@tok@cm\endcsname{\let\PY@it=\textit\def\PY@tc##1{\textcolor[rgb]{0.25,0.50,0.50}{##1}}}
\expandafter\def\csname PY@tok@cpf\endcsname{\let\PY@it=\textit\def\PY@tc##1{\textcolor[rgb]{0.25,0.50,0.50}{##1}}}
\expandafter\def\csname PY@tok@c1\endcsname{\let\PY@it=\textit\def\PY@tc##1{\textcolor[rgb]{0.25,0.50,0.50}{##1}}}
\expandafter\def\csname PY@tok@cs\endcsname{\let\PY@it=\textit\def\PY@tc##1{\textcolor[rgb]{0.25,0.50,0.50}{##1}}}

\def\PYZbs{\char`\\}
\def\PYZus{\char`\_}
\def\PYZob{\char`\{}
\def\PYZcb{\char`\}}
\def\PYZca{\char`\^}
\def\PYZam{\char`\&}
\def\PYZlt{\char`\<}
\def\PYZgt{\char`\>}
\def\PYZsh{\char`\#}
\def\PYZpc{\char`\%}
\def\PYZdl{\char`\$}
\def\PYZhy{\char`\-}
\def\PYZsq{\char`\'}
\def\PYZdq{\char`\"}
\def\PYZti{\char`\~}
% for compatibility with earlier versions
\def\PYZat{@}
\def\PYZlb{[}
\def\PYZrb{]}
\makeatother


    % Exact colors from NB
    \definecolor{incolor}{rgb}{0.0, 0.0, 0.5}
    \definecolor{outcolor}{rgb}{0.545, 0.0, 0.0}



    
    % Prevent overflowing lines due to hard-to-break entities
    \sloppy 
    % Setup hyperref package
    \hypersetup{
      breaklinks=true,  % so long urls are correctly broken across lines
      colorlinks=true,
      urlcolor=urlcolor,
      linkcolor=linkcolor,
      citecolor=citecolor,
      }
    % Slightly bigger margins than the latex defaults
    
    \geometry{verbose,tmargin=1in,bmargin=1in,lmargin=1in,rmargin=1in}
    
    

    \begin{document}
    
    
    \maketitle
    
    

    
    \subsubsection{Travis Chen}\label{travis-chen}

    \subsection{Introduction}\label{introduction}

    A widely stated (and disputed) result is the "10 days" rule --- that if
an investor were to exclude the best 10 days in market history then the
stock market would be a horrible investment, and conversely, excluding
the worst 10 days in market history would make it fantastic. Seyhun
{[}1{]} visualized this result with various levels of outlier exclusion:

    

    Regardless of the degree to which such a result is overstated or relies
on faulty assumptions, it is clear that evaluating tail end risks (and
rewards) are crucial to understanding the success of financial
strategies and institutions. Such "Black Swan" or "Tail End" events may
be responsible for the vast majority of returns and losses on a given
strategy.

In this notebook, I hope to illustrate some examples of how tail end
risk can affect investment strategies the way they're traditionally
understood, particularly with regards to signal usage and hedge funds.
In particular, I will touch upon how the hedge fund payout structure
will even

    \subsection{Why is it hard to predict tail ends (using signals)? Or,
"Correlation
Shmorrelation"}\label{why-is-it-hard-to-predict-tail-ends-using-signals-or-correlation-shmorrelation}

    Assume we have a fund, Tail Capital that focuses on predicting tail end
events. It develops a signal \(X\) that has above \(\rho_{threshold}\)
correlation on back-tested returns \(Y\), and then trades in proportion
to the strength of that signal.

    As a few simplifying assumptions, let: 1. The signal \(X\) is normally
distributed with 0 mean and standard deviation 1: \(X \sim N(0, 1)\) 2.
The daily return \(Y\) is additive and also normally distributed with 0
mean and standard deviation 1: \(Y \sim N(0, 1)\) 3. \(X\) and \(Y\)
have a correlation of \(\rho\), that is:
\(\frac{E[(X - \mu_{X})(Y - \mu_{Y})]}{\sigma_{X} \sigma_{Y}} = E[X Y] = 1\)

    \textbf{(Note that this normality assumption often used in financial
models may also itself be violated by certain tail risk challenges, but
we will ignore this for now.)}

    We wish to assess the conditional probability that \(Y > K\) given that
\(X > K\), that is:

\(P(Y > K | X > K) = \frac{P(X > K, Y > K)}{P(X > K)}\)

Another interpretation of this expression is that this is the ratio of
the probability of both \(X\) and \(Y\) exceeding some threshold \(K\)
assuming correlation \(\rho\) over the probability of both \(X\) and
\(Y\) exceeding \(K\) assuming correlation \(1\) (given that if the
correlation is 1, then \(P(X > K, Y > K) = P(X > K)\). Taleb calls this
the "proportion of certainty."

    The evaluation of this expression (Taleb {[}2{]}) is performed as
follows:

    

    The interpretation here is that for tail end events, i.e. when \(K\) is
many standard deviations away from \(0\), a signal is almost meaningless
in terms of telling us the odds that the output variable returns, \(Y\),
are also greater than \(K\) unless the correlation \(\rho\) is very
close to 1.

In the words of Taleb, "Correlation between X and Y carries
disproportionate information for the ordinary, and practically no
information for the tails." As an illuminating example, assuming IQ is
the signal and intelligence is the output, one would need something
resembling \(\rho > 0.98\) to 'explain' genius with greater than 50\%
confidence.

    \subsection{A Simulated Example}\label{a-simulated-example}

    Assume Tail Capital uses a highly leveraged strategy to capitalize on
tail end predictions (i.e. by buying out of money puts). We evaluate, at
different levels of correlation, what are the actual odds that Tail
Capital actually lands on its predictions.

Assume a naive strategy of buying the OOM option corresponding a
thresholded signal (i.e. if our threshold is 3\(\sigma\) and the signal
crosses 3\(\sigma\), then buy the put at 3\(\sigma\)). We evaluate the
probability that our option reaches the exercise price at various level
of \(\sigma\) and \(\rho\).

We first sample our priors from a \(N(0, 1)\) distribution, then
multiply our \((X, Y)\) matrix by the Cholesky factorization of the
correlation matrix to embed the correlational dependence.

    \begin{Verbatim}[commandchars=\\\{\}]
{\color{incolor}In [{\color{incolor}49}]:} \PY{o}{\PYZpc{}}\PY{k}{matplotlib} inline
\end{Verbatim}


    \begin{Verbatim}[commandchars=\\\{\}]
{\color{incolor}In [{\color{incolor}61}]:} \PY{k+kn}{import} \PY{n+nn}{numpy} \PY{k}{as} \PY{n+nn}{np}
         \PY{k+kn}{from} \PY{n+nn}{scipy}\PY{n+nn}{.}\PY{n+nn}{linalg} \PY{k}{import} \PY{n}{eigh}\PY{p}{,} \PY{n}{cholesky}
         \PY{k+kn}{import} \PY{n+nn}{matplotlib}\PY{n+nn}{.}\PY{n+nn}{pyplot} \PY{k}{as} \PY{n+nn}{plt}
         
         \PY{n}{np}\PY{o}{.}\PY{n}{random}\PY{o}{.}\PY{n}{seed}\PY{p}{(}\PY{l+m+mi}{42}\PY{p}{)}
         
         \PY{n}{mu} \PY{o}{=} \PY{l+m+mi}{0}
         \PY{n}{sigma} \PY{o}{=} \PY{l+m+mi}{1}
         \PY{n}{num\PYZus{}trading\PYZus{}days} \PY{o}{=} \PY{l+m+mi}{100000}
         \PY{n}{std\PYZus{}vals} \PY{o}{=} \PY{p}{[}\PY{l+m+mf}{0.5}\PY{p}{,} \PY{l+m+mi}{1}\PY{p}{,} \PY{l+m+mf}{1.5}\PY{p}{,} \PY{l+m+mi}{2}\PY{p}{,} \PY{l+m+mf}{2.5}\PY{p}{,} \PY{l+m+mi}{3}\PY{p}{,} \PY{l+m+mf}{3.5}\PY{p}{,} \PY{l+m+mi}{4}\PY{p}{,} \PY{l+m+mf}{4.5}\PY{p}{,} \PY{l+m+mi}{5}\PY{p}{]}
         
         \PY{c+c1}{\PYZsh{} Sampling various values of correlation, rho}
         \PY{k}{for} \PY{n}{rho} \PY{o+ow}{in} \PY{p}{[}\PY{l+m+mf}{0.4}\PY{p}{,} \PY{l+m+mf}{0.6}\PY{p}{,} \PY{l+m+mf}{0.8}\PY{p}{,} \PY{l+m+mf}{0.99}\PY{p}{]}\PY{p}{:}
             \PY{n}{std\PYZus{}signal\PYZus{}landing\PYZus{}pct} \PY{o}{=} \PY{p}{[}\PY{p}{]}
             
             \PY{c+c1}{\PYZsh{} Sampling various standard deviations to try to predict the output variable at}
             \PY{k}{for} \PY{n}{n\PYZus{}std} \PY{o+ow}{in} \PY{n}{std\PYZus{}vals}\PY{p}{:}
                 \PY{c+c1}{\PYZsh{} Initialize the \PYZdq{}prior\PYZdq{} of X, Y to be a normal distribution centered around 0 with standard deviation 1}
                 \PY{n}{M} \PY{o}{=} \PY{n}{np}\PY{o}{.}\PY{n}{random}\PY{o}{.}\PY{n}{normal}\PY{p}{(}\PY{n}{mu}\PY{p}{,} \PY{n}{sigma}\PY{p}{,} \PY{p}{(}\PY{n}{num\PYZus{}trading\PYZus{}days}\PY{p}{,} \PY{l+m+mi}{2}\PY{p}{)}\PY{p}{)}
         
                 \PY{c+c1}{\PYZsh{} Use the Cholesky Factorization of R, the correlation matrix, to sample X and Y}
                 \PY{n}{R} \PY{o}{=} \PY{n}{np}\PY{o}{.}\PY{n}{array}\PY{p}{(}\PY{p}{[}\PY{p}{[}\PY{l+m+mi}{1}\PY{p}{,} \PY{n}{rho}\PY{p}{]}\PY{p}{,} \PY{p}{[}\PY{n}{rho}\PY{p}{,} \PY{l+m+mi}{1}\PY{p}{]}\PY{p}{]}\PY{p}{)}\PY{p}{;}
                 \PY{n}{L} \PY{o}{=} \PY{n}{cholesky}\PY{p}{(}\PY{n}{R}\PY{p}{,} \PY{n}{lower}\PY{o}{=}\PY{k+kc}{True}\PY{p}{)}
                 \PY{n}{M} \PY{o}{=} \PY{n}{np}\PY{o}{.}\PY{n}{dot}\PY{p}{(}\PY{n}{M}\PY{p}{,} \PY{n}{L}\PY{p}{)}
                 \PY{n}{X} \PY{o}{=} \PY{n}{M}\PY{p}{[}\PY{p}{:}\PY{p}{,} \PY{l+m+mi}{0}\PY{p}{]}
                 \PY{n}{Y} \PY{o}{=} \PY{n}{M}\PY{p}{[}\PY{p}{:}\PY{p}{,} \PY{l+m+mi}{1}\PY{p}{]}
         
                 \PY{c+c1}{\PYZsh{} What percent of the time does our signal at n std out actually predict a price at n std out?}
                 \PY{n}{num\PYZus{}signal\PYZus{}fired} \PY{o}{=} \PY{l+m+mi}{0}
                 \PY{n}{num\PYZus{}correct\PYZus{}predictions} \PY{o}{=} \PY{l+m+mi}{0}
                 \PY{k}{for} \PY{p}{(}\PY{n}{x}\PY{p}{,} \PY{n}{y}\PY{p}{)} \PY{o+ow}{in} \PY{n+nb}{zip}\PY{p}{(}\PY{n}{X}\PY{p}{,} \PY{n}{Y}\PY{p}{)}\PY{p}{:}
                     \PY{k}{if} \PY{n}{x} \PY{o}{\PYZgt{}} \PY{n}{n\PYZus{}std}\PY{p}{:}
                         \PY{n}{num\PYZus{}signal\PYZus{}fired} \PY{o}{+}\PY{o}{=} \PY{l+m+mi}{1}
                         \PY{k}{if} \PY{n}{y} \PY{o}{\PYZgt{}} \PY{n}{n\PYZus{}std}\PY{p}{:}
                             \PY{n}{num\PYZus{}correct\PYZus{}predictions} \PY{o}{+}\PY{o}{=} \PY{l+m+mi}{1}
                     \PY{k}{elif} \PY{n}{x} \PY{o}{\PYZlt{}} \PY{o}{\PYZhy{}}\PY{n}{n\PYZus{}std}\PY{p}{:}
                         \PY{n}{num\PYZus{}signal\PYZus{}fired} \PY{o}{+}\PY{o}{=} \PY{l+m+mi}{1}
                         \PY{k}{if} \PY{n}{y} \PY{o}{\PYZlt{}} \PY{o}{\PYZhy{}}\PY{n}{n\PYZus{}std}\PY{p}{:}
                             \PY{n}{num\PYZus{}correct\PYZus{}predictions} \PY{o}{+}\PY{o}{=} \PY{l+m+mi}{1}
                             
                 \PY{c+c1}{\PYZsh{} Append the number of times, at correlation rho, our signal thresholded at std\PYZus{}val}
                 \PY{c+c1}{\PYZsh{} correctly predicts the price at least the same number of standard deviations out}
                 \PY{k}{if} \PY{n}{num\PYZus{}signal\PYZus{}fired} \PY{o}{==} \PY{l+m+mi}{0}\PY{p}{:}
                     \PY{n}{std\PYZus{}signal\PYZus{}landing\PYZus{}pct}\PY{o}{.}\PY{n}{append}\PY{p}{(}\PY{l+m+mi}{0}\PY{p}{)}
                 \PY{k}{else}\PY{p}{:}
                     \PY{n}{std\PYZus{}signal\PYZus{}landing\PYZus{}pct}\PY{o}{.}\PY{n}{append}\PY{p}{(}\PY{n}{num\PYZus{}correct\PYZus{}predictions} \PY{o}{/} \PY{n}{num\PYZus{}signal\PYZus{}fired}\PY{p}{)}
             
             \PY{n}{plt}\PY{o}{.}\PY{n}{plot}\PY{p}{(}\PY{n}{std\PYZus{}vals}\PY{p}{,} \PY{n}{std\PYZus{}signal\PYZus{}landing\PYZus{}pct}\PY{p}{)}
         \PY{n}{plt}\PY{o}{.}\PY{n}{xlabel}\PY{p}{(}\PY{l+s+s1}{\PYZsq{}}\PY{l+s+s1}{\PYZsh{} Std Dev Signal Threshold}\PY{l+s+s1}{\PYZsq{}}\PY{p}{)}
         \PY{n}{plt}\PY{o}{.}\PY{n}{ylabel}\PY{p}{(}\PY{l+s+s1}{\PYZsq{}}\PY{l+s+s1}{Probability of Reaching Threshold Price}\PY{l+s+s1}{\PYZsq{}}\PY{p}{)}
         \PY{n}{plt}\PY{o}{.}\PY{n}{title}\PY{p}{(}\PY{l+s+s1}{\PYZsq{}}\PY{l+s+s1}{Probability of Reaching Threshold Price @ Std Dev Threshold}\PY{l+s+s1}{\PYZsq{}}\PY{p}{)}
             
\end{Verbatim}


\begin{Verbatim}[commandchars=\\\{\}]
{\color{outcolor}Out[{\color{outcolor}61}]:} Text(0.5,1,'Probability of Reaching Threshold Price @ Std Dev Threshold')
\end{Verbatim}
            
    \begin{center}
    \adjustimage{max size={0.9\linewidth}{0.9\paperheight}}{output_16_1.png}
    \end{center}
    { \hspace*{\fill} \\}
    
    The interpretation here is that once we try to predict our output
variable, the price, at more than 3 standard deviations out, our signal
becomes practically useless. The decay of our signal corresponding to
the tail end also appears to be exponential.

    \subsection{Why Hedge Funds are incentivized to practice Bad Tail end
Risk
Management?}\label{why-hedge-funds-are-incentivized-to-practice-bad-tail-end-risk-management}

    Simply put, this is because hedge funds are paid annually on both a flat
and commission fee basis, and do not symmetrically suffer the downside
of losing their investors money (beyond firm 'blowup' at a certain
threshold). Thus, a hedge fund manager may be incentivized to make
asymmetric high probability of return but huge downside bets, thus
becoming an optimization problem of maximizing \emph{time} to losses
rather than maximizing \emph{returns}. As mentioned in class, they may
be incentivized to simply sell very out of money options every year and
hope that by the time there's a year that the strike prices hit, they're
retired by then.

    \subsection{A Contrived Example}\label{a-contrived-example}

    Let us simulate hedge fund managers as operating under the following
(made-up) incentive structure: 1. They get paid a flat rate of 1 dollar
a year. 2. They get paid a 20\% commission of their net profit. 3. The
fund shuts down if it has a year where it loses 10 dollars (it will
shoulder 20\% of this loss). The fund continuously runs until it gets
shut down.

    To immediately see an issue with such a (contrived) incentive structure,
consider 2 different (contrived) yearly strategies.

\begin{enumerate}
\def\labelenumi{\Alph{enumi})}
\tightlist
\item
  99\% of gaining +10, 1\% chance of losing for -1000\\
\item
  80\% chance of gaining +10, 20\% chance of losing for -10
\end{enumerate}

    We can calculate that the yearly expected value for A is:\\
\(0.99 * 10 - 0.01 * 1000 = -0.1\)

And for B is:\\
\(0.8 * 10 - 0.2 * 10 = +6\)

It is fairly obvious that strategy A has higher expected returns thn
strategy B. But, what is the expected payout to a hedge fund manager?

    We can recursively calculate this expected utility based on the fact
that if one of these strategies has a winning year, then it is as if the
manager starts again fresh from the next year, with the payout from the
previous year in the bank.

    \(E[Strategy] = Flat Rate + Prob Winning Year (Commission Rate * Winning Payout + E[Strategy]) + Prob Losing Year * (Commision Rate * Losing Payout)\)

\(E[A] = 1 + 0.99 * ((0.2 * 10) + E[A]) + 0.01 * (0.2 * -1000)\)\\
\(E[A] = +98\)

\(E[B] = 1 + 0.8 * ((0.2 * 10) + E[B]) + 0.2 * (0.2 * -10)\)\\
\(E[B] = +11\)

    Although Strategy A is far more risky (in terms of maximum possible
loss) AND has lower expected value to investors, it has much higher
expected payout to the hedge fund manager!

    \subsection{Simulated Investment Hedge Fund
Strategies}\label{simulated-investment-hedge-fund-strategies}

    Let us define an investment strategy as a yearly distribution (summing
to 1) of probability of integer payouts in the range \([-100, 100]\). We
will simulate according to the same rules above, and evaluate both the
expected value of a strategy and the expected utility to a hedge fund
manager. We will also add a constraint that a fund will never exist for
more than 100 years.

We will sample the strategies, seeing the relationship between the
expected value of the strategy and hedge fund manager payout; and see if
the skewness (i.e. existence of tails) in the strategy has any
relationship with payout above expectation.

    \begin{Verbatim}[commandchars=\\\{\}]
{\color{incolor}In [{\color{incolor}114}]:} \PY{k+kn}{import} \PY{n+nn}{scipy}\PY{n+nn}{.}\PY{n+nn}{stats}
          
          \PY{n}{max\PYZus{}returns} \PY{o}{=} \PY{l+m+mi}{100}
          
          \PY{c+c1}{\PYZsh{} Gets expected value from relative weights by performing \PYZhy{}20 * P(return of \PYZhy{}20) + \PYZhy{}19 * P(return of \PYZhy{}19, etc.) + ...}
          \PY{k}{def} \PY{n+nf}{get\PYZus{}expected\PYZus{}value\PYZus{}from\PYZus{}relative\PYZus{}weights}\PY{p}{(}\PY{n}{relative\PYZus{}weights}\PY{p}{)}\PY{p}{:}
              \PY{k}{return} \PY{n+nb}{sum}\PY{p}{(}\PY{n}{i} \PY{o}{*} \PY{n}{w} \PY{k}{for} \PY{n}{i}\PY{p}{,} \PY{n}{w} \PY{o+ow}{in} \PY{n+nb}{zip}\PY{p}{(}\PY{p}{[}\PY{n}{i} \PY{k}{for} \PY{n}{i} \PY{o+ow}{in} \PY{n+nb}{range}\PY{p}{(}\PY{o}{\PYZhy{}}\PY{n}{max\PYZus{}returns}\PY{p}{,} \PY{n}{max\PYZus{}returns}\PY{o}{+}\PY{l+m+mi}{1}\PY{p}{)}\PY{p}{]}\PY{p}{,} \PY{n}{relative\PYZus{}weights}\PY{p}{)}\PY{p}{)}
          
          \PY{k}{def} \PY{n+nf}{simulate\PYZus{}investment\PYZus{}fund\PYZus{}payout}\PY{p}{(}\PY{n}{relative\PYZus{}weights}\PY{p}{)}\PY{p}{:}
              \PY{n}{payout} \PY{o}{=} \PY{l+m+mi}{0}
              
              \PY{c+c1}{\PYZsh{} Maximum of 100 years}
              \PY{k}{for} \PY{n}{i} \PY{o+ow}{in} \PY{n+nb}{range}\PY{p}{(}\PY{l+m+mi}{100}\PY{p}{)}\PY{p}{:}
                  \PY{n}{payout} \PY{o}{+}\PY{o}{=} \PY{l+m+mi}{1} \PY{c+c1}{\PYZsh{} Add 1, the fixed management fee }
                  
                  \PY{c+c1}{\PYZsh{} Sample yearly return from [\PYZhy{}20, 20] based on the relative weights distribution }
                  \PY{n}{yearly\PYZus{}return} \PY{o}{=} \PY{n}{np}\PY{o}{.}\PY{n}{random}\PY{o}{.}\PY{n}{choice}\PY{p}{(}\PY{p}{[}\PY{n}{i} \PY{k}{for} \PY{n}{i} \PY{o+ow}{in} \PY{n+nb}{range}\PY{p}{(}\PY{o}{\PYZhy{}}\PY{n}{max\PYZus{}returns}\PY{p}{,} \PY{n}{max\PYZus{}returns}\PY{o}{+}\PY{l+m+mi}{1}\PY{p}{)}\PY{p}{]}\PY{p}{,} \PY{l+m+mi}{1}\PY{p}{,} \PY{n+nb}{list}\PY{p}{(}\PY{n}{relative\PYZus{}weights}\PY{p}{)}\PY{p}{)}\PY{p}{[}\PY{l+m+mi}{0}\PY{p}{]}
                  
                  \PY{c+c1}{\PYZsh{} Add the 20\PYZpc{} commission}
                  \PY{n}{payout} \PY{o}{+}\PY{o}{=} \PY{l+m+mf}{0.2} \PY{o}{*} \PY{n}{yearly\PYZus{}return}
                  
                  \PY{c+c1}{\PYZsh{} If it has a \PYZhy{}10 or worse losing year, the fund ends}
                  \PY{k}{if} \PY{n}{payout} \PY{o}{\PYZlt{}}\PY{o}{=} \PY{o}{\PYZhy{}}\PY{l+m+mi}{10}\PY{p}{:}
                      \PY{k}{return} \PY{n}{payout}
              \PY{k}{return} \PY{n}{payout}
          
          \PY{n}{expected\PYZus{}values} \PY{o}{=} \PY{p}{[}\PY{p}{]}
          \PY{n}{average\PYZus{}payouts} \PY{o}{=} \PY{p}{[}\PY{p}{]}
          \PY{n}{skewness} \PY{o}{=} \PY{p}{[}\PY{p}{]}
          
          \PY{k}{for} \PY{n}{i} \PY{o+ow}{in} \PY{n+nb}{range}\PY{p}{(}\PY{l+m+mi}{100}\PY{p}{)}\PY{p}{:}
              \PY{c+c1}{\PYZsh{} Get relative weights from each [\PYZhy{}20, 20] value from a uniform [0, 1] distribution,}
              \PY{c+c1}{\PYZsh{} such that they sum to 1}
              \PY{n}{relative\PYZus{}weights} \PY{o}{=} \PY{n}{np}\PY{o}{.}\PY{n}{random}\PY{o}{.}\PY{n}{random}\PY{p}{(}\PY{l+m+mi}{2}\PY{o}{*}\PY{n}{max\PYZus{}returns} \PY{o}{+} \PY{l+m+mi}{1}\PY{p}{)}
              \PY{n}{relative\PYZus{}weights} \PY{o}{/}\PY{o}{=} \PY{n}{np}\PY{o}{.}\PY{n}{sum}\PY{p}{(}\PY{n}{relative\PYZus{}weights}\PY{p}{)}
              
              \PY{c+c1}{\PYZsh{} Calculate expected value for investors from the randomly sampled investment strategy}
              \PY{n}{ev} \PY{o}{=} \PY{n}{get\PYZus{}expected\PYZus{}value\PYZus{}from\PYZus{}relative\PYZus{}weights}\PY{p}{(}\PY{n}{relative\PYZus{}weights}\PY{p}{)}
              \PY{n}{expected\PYZus{}values}\PY{o}{.}\PY{n}{append}\PY{p}{(}\PY{n}{ev}\PY{p}{)}
              
              \PY{c+c1}{\PYZsh{} Calculated expected payotut for hedge fund managers}
              \PY{n}{simulated\PYZus{}payouts} \PY{o}{=} \PY{p}{[}\PY{p}{]}
              \PY{k}{for} \PY{n}{j} \PY{o+ow}{in} \PY{n+nb}{range}\PY{p}{(}\PY{l+m+mi}{10}\PY{p}{)}\PY{p}{:}
                  \PY{n}{simulated\PYZus{}payouts}\PY{o}{.}\PY{n}{append}\PY{p}{(}\PY{n}{simulate\PYZus{}investment\PYZus{}fund\PYZus{}payout}\PY{p}{(}\PY{n}{relative\PYZus{}weights}\PY{p}{)}\PY{p}{)}
                  
              \PY{n}{average\PYZus{}payouts}\PY{o}{.}\PY{n}{append}\PY{p}{(}\PY{n}{np}\PY{o}{.}\PY{n}{mean}\PY{p}{(}\PY{n}{simulated\PYZus{}payouts}\PY{p}{)}\PY{p}{)}
               
              \PY{c+c1}{\PYZsh{} Calculate the skewness of the investment strategy distribution}
              \PY{n}{skewness}\PY{o}{.}\PY{n}{append}\PY{p}{(}\PY{n}{scipy}\PY{o}{.}\PY{n}{stats}\PY{o}{.}\PY{n}{skew}\PY{p}{(}\PY{n}{relative\PYZus{}weights}\PY{p}{)}\PY{p}{)}
\end{Verbatim}


    \begin{Verbatim}[commandchars=\\\{\}]
{\color{incolor}In [{\color{incolor}122}]:} \PY{n}{x} \PY{o}{=} \PY{n}{expected\PYZus{}values}
          \PY{n}{y} \PY{o}{=} \PY{n}{average\PYZus{}payouts}
          \PY{n}{plt}\PY{o}{.}\PY{n}{scatter}\PY{p}{(}\PY{n}{x}\PY{p}{,} \PY{n}{y}\PY{p}{)}
          \PY{n}{plt}\PY{o}{.}\PY{n}{xlabel}\PY{p}{(}\PY{l+s+s1}{\PYZsq{}}\PY{l+s+s1}{Yearly Expected Value}\PY{l+s+s1}{\PYZsq{}}\PY{p}{)}
          \PY{n}{plt}\PY{o}{.}\PY{n}{ylabel}\PY{p}{(}\PY{l+s+s1}{\PYZsq{}}\PY{l+s+s1}{Hedge Fund Manager Payout}\PY{l+s+s1}{\PYZsq{}}\PY{p}{)}
          \PY{n}{plt}\PY{o}{.}\PY{n}{title}\PY{p}{(}\PY{l+s+s1}{\PYZsq{}}\PY{l+s+s1}{Hedge Fund Manager Payout vs. Strategy Yearly Expected Value}\PY{l+s+s1}{\PYZsq{}}\PY{p}{)}
\end{Verbatim}


\begin{Verbatim}[commandchars=\\\{\}]
{\color{outcolor}Out[{\color{outcolor}122}]:} Text(0.5,1,'Hedge Fund Manager Payout vs. Strategy Yearly Expected Value')
\end{Verbatim}
            
    \begin{center}
    \adjustimage{max size={0.9\linewidth}{0.9\paperheight}}{output_30_1.png}
    \end{center}
    { \hspace*{\fill} \\}
    
    Here we see that the expected value of a strategy in this setup has a
weak at best relationship with how much a hedge fund manager can expect
to be paid from the strategy.

    We use skewness as a proxy for how fat the tails of the strategy return
distribution are. Mathematically, skewness is defined as:

    

    \begin{Verbatim}[commandchars=\\\{\}]
{\color{incolor}In [{\color{incolor}129}]:} \PY{n}{plt}\PY{o}{.}\PY{n}{scatter}\PY{p}{(}\PY{n}{skewness}\PY{p}{,} \PY{n}{average\PYZus{}payouts}\PY{p}{)}
          \PY{n}{plt}\PY{o}{.}\PY{n}{xlabel}\PY{p}{(}\PY{l+s+s1}{\PYZsq{}}\PY{l+s+s1}{Strategy Returns Skewness}\PY{l+s+s1}{\PYZsq{}}\PY{p}{)}
          \PY{n}{plt}\PY{o}{.}\PY{n}{ylabel}\PY{p}{(}\PY{l+s+s1}{\PYZsq{}}\PY{l+s+s1}{Hedge Fund Manager Payout}\PY{l+s+s1}{\PYZsq{}}\PY{p}{)}
          \PY{n}{plt}\PY{o}{.}\PY{n}{title}\PY{p}{(}\PY{l+s+s1}{\PYZsq{}}\PY{l+s+s1}{Hedge Fund Manager Payout vs. Skewness}\PY{l+s+s1}{\PYZsq{}}\PY{p}{)}
\end{Verbatim}


\begin{Verbatim}[commandchars=\\\{\}]
{\color{outcolor}Out[{\color{outcolor}129}]:} Text(0.5,1,'Hedge Fund Manager Payout vs. Skewness')
\end{Verbatim}
            
    \begin{center}
    \adjustimage{max size={0.9\linewidth}{0.9\paperheight}}{output_34_1.png}
    \end{center}
    { \hspace*{\fill} \\}
    
    Note that now get a much tighter relationship between investment
strategy skewness to hedge fund manager payout than yearly expected
value! This implies that under our contrived setup, \textbf{hedge fund
managers are incentivized to pursue strategy with more tail end risk},
to "accumulate bonuses before eventual "blowup" for which he does not
have to repay previous compensation {[}and{]} make a series of
asymmetric bets (high probability of small profits, small probability of
large losses) below their probabilistic fair value" (Taleb {[}3{]})

    \subsection{Conclusions}\label{conclusions}

    We have demonstrated that\\
1. Using correlation via signals may be inappropriate for identifying
tail end returns\\
2. Hedge fund managers have perverse incentives to maintain high
probability of profit even at the risk of extreme tail end downside

The solution to the problem is unclear, but likely lies in a realignment
of incentives to \emph{make risk symmetric}, that is ensure that hedge
fund managers share in the downside of their investments. As Taleb
{[}3{]} articulates, "The captain goes down with the ship; all captains
and all ships: making everyone involved in riskbearing accountable, no
exception, not a single one --- morally, legally, whatever can be done."

    \subsection{References}\label{references}

    \begin{enumerate}
\def\labelenumi{\arabic{enumi}.}
\tightlist
\item
  STOCK MARKET EXTREMES AND PORTFOLIO PERFORMANCE. A study commissioned
  by Towneley Capital Management and conducted by Professor H. Nejat
  Seyhun, University of Michigan. 1993.
  http://www.towneley.com/wp-content/uploads/2016/01/Research.-TCM-Mkt-Timing-Study-1993.pdf
\item
  Nassim Taleb's Twitter.
  https://twitter.com/nntaleb/status/1135116646442590208. 2019
\item
  "Why did the Crisis of 2008 Happen?" Nassim Taleb. 2010.
\end{enumerate}


    % Add a bibliography block to the postdoc
    
    
    
    \end{document}
